\documentclass[a4paper,zihao=5,UTF8]{ctexart}
\usepackage[top=2.3cm,bottom=2cm,left=1.7cm,right=1.7cm]{geometry} 
\usepackage{amsmath, amssymb}
\usepackage{color}
\usepackage{hyperref} 
\usepackage{pythonhighlight}
\usepackage{listings}
\usepackage{mathrsfs} 
\usepackage{booktabs}
\usepackage{amsthm}
\usepackage{longtable} 
\usepackage{graphicx}
\usepackage{subfigure}
\usepackage{caption}
\usepackage{fontspec}
\usepackage{titlesec}
\usepackage{fancyhdr}
\usepackage{latexsym}
\usepackage{subfigure}
\usepackage{braket}
\usepackage{cite}
\usepackage[version=4]{mhchem}

\CTEXsetup[format={\Large\bfseries}]{section}
\def\d{\mathrm{d}}
\def\e{\mathrm{e}}
\def\i{\mathrm{i}}
\def\dps{\displaystyle}
\newcommand{\mr}[1]{\mathrm{#1}}
\newcommand{\mb}[1]{\mathbf{#1}}
\newcommand{\dv}[2]{\frac{\d{#1}}{\d{#2}}}
\newcommand{\pdv}[2]{\frac{\partial{#1}}{\partial{#2}}}
\def\degree{$^{\circ}$}
\def\celsius{^{\circ}\mr{C}}
\title{\textbf{实验五 用对-叔丁基杯[8]芳烃分离\ce{C_{60}}\cite{inorganic_chemistry_1}}}
\author{王崇斌\;1800011716}
\makeatletter
\makeatother
\begin{document}
	\pagestyle{fancy}
	\pagestyle{fancy}
    \lhead{无机化学实验}
	\chead{}
	\rhead{\today}
	\maketitle
    \thispagestyle{fancy}
    \section{实验目的}
    \begin{enumerate}
        \item 了解超分子化学在化学分离中的应用
        \item 了解杯芳烃的合成
        \item 了解杯芳烃分离\ce{C_{60}}与\ce{C_{70}}的原理与方法
    \end{enumerate}
	\section{实验原理}
    \subsection{杯芳烃的合成}
    \subsection{杯芳烃选择性分离\ce{C_{60}}与\ce{C_{70}}的原理}
    \section{实验内容}

    \subsection{从富勒烯混合物中分离和纯化\ce{C_{60}}}
    \subsection{测定产品\ce{C_{60}}的纯度与含量}
    \section{结果与讨论}
    \subsection{计算\ce{C_{60}}甲苯溶液中的\ce{C_{60}}含量和收率}
    \subsection{讨论加热回流和超声在对-叔丁基杯[8]芳烃分离\ce{C_{60}}与\ce{C_{70}}
    }
    \section{思考题}

    \bibliographystyle{plain}
	\bibliography{ref}
\end{document}