\documentclass[a4paper,zihao=5,UTF8]{ctexart}
\usepackage[top=2.3cm,bottom=2cm,left=1.7cm,right=1.7cm]{geometry} 
\usepackage{amsmath, amssymb}
\usepackage{color}
\usepackage{hyperref} 
\usepackage{pythonhighlight}
\usepackage{listings}
\usepackage{mathrsfs} 
\usepackage{booktabs}
\usepackage{amsthm}
\usepackage{longtable} 
\usepackage{graphicx}
\usepackage{subfigure}
\usepackage{caption}
\usepackage{fontspec}
\usepackage{titlesec}
\usepackage{fancyhdr}
\usepackage{latexsym}
\usepackage{subfigure}
\usepackage{braket}
\usepackage{cite}
\usepackage[version=4]{mhchem}
\CTEXsetup[format={\Large\bfseries}]{section}
\def\d{\mathrm{d}}
\def\e{\mathrm{e}}
\def\i{\mathrm{i}}
\def\dps{\displaystyle}
\newcommand{\mr}[1]{\mathrm{#1}}
\newcommand{\mb}[1]{\mathbf{#1}}
\newcommand{\dv}[2]{\frac{\d{#1}}{\d{#2}}}
\newcommand{\pdv}[2]{\frac{\partial{#1}}{\partial{#2}}}
\def\degree{$^{\circ}$}
\title{\textbf{实验一 \ce{SrO}-\ce{Al2O3}二元体系中几种荧光材料的合成和表征\cite{inorganic_chemistry_1}}}
\author{王崇斌\;1800011716}
\makeatletter
\makeatother
\begin{document}
	\pagestyle{fancy}
	\pagestyle{fancy}
    \lhead{无机化学实验}
	\chead{}
	\rhead{\today}
	\maketitle
    \thispagestyle{fancy}
	\section{实验目的}
	\begin{enumerate}
		\item 学习高温固相合成的基本方法
		\item 学习软化学制备前驱体的基本方法
		\item 了解固体荧光材料的发光原理和基本表征方法
	\end{enumerate}
	\section{实验原理}
	\subsection{高温固相合成}
	高温固相合成通常用于合成无机固体材料。反应中存在多个固体物相,反应主要
	发生在固态反应物的接触面上,同时在其上形成产物层;然后反应物通过扩散作用跨过产物层
	进一步反应。由于固相难以混合充分,反应物颗粒之间的接触面积大小受反应物颗粒直径影响明显
	,同时由于固体中原子扩散速率远低于液相或气相(晶格比较稳定,常温下只会在平衡位置
	附近振动),因此固相化学反应通常需要在高温下进行(增加固体中原子的扩散速率,甚至
	熔化反应物或者产物以达到充分混合的目的),反应时间较长,难以得到高纯度、均匀的、物相单一
	的产物。dengg
	\par 
	通常影响固相化学反应速率的因素都有:反应温度(这个前面讨论了),反应物混合的均匀程度(决定了
	反应物之间的接触面积),反应物物相(不同的物相有着不同的表面能和稳定性),添加助溶剂(提高
	反应物表面离子的扩散速率),等等。
	\subsection{软化学制备前驱体}
	
	\subsection{荧光材料的发光原理}
	\section{实验操作步骤}
	\subsection{高温固相反应合成荧光材料}
	\subsection{燃烧法合成荧光材料前驱体}
	\section{荧光材料的表征}
	\subsection{发光性能的表征}
	\subsection{物相的鉴定}

	
	\bibliographystyle{plain}
	\bibliography{ref}
\end{document}