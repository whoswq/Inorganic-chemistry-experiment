\documentclass[a4paper,zihao=5,UTF8]{ctexart}
\usepackage[top=2.3cm,bottom=2cm,left=1.7cm,right=1.7cm]{geometry} 
\usepackage{amsmath, amssymb}
\usepackage{color}
\usepackage{hyperref} 
\usepackage{pythonhighlight}
\usepackage{listings}
\usepackage{mathrsfs} 
\usepackage{booktabs}
\usepackage{amsthm}
\usepackage{longtable} 
\usepackage{graphicx}
\usepackage{subfigure}
\usepackage{caption}
\usepackage{fontspec}
\usepackage{titlesec}
\usepackage{fancyhdr}
\usepackage{latexsym}
\usepackage{subfigure}
\usepackage{braket}
\usepackage{cite}
\usepackage[version=4]{mhchem}
\CTEXsetup[format={\Large\bfseries}]{section}
\def\d{\mathrm{d}}
\def\e{\mathrm{e}}
\def\i{\mathrm{i}}
\def\dps{\displaystyle}
\newcommand{\mr}[1]{\mathrm{#1}}
\newcommand{\mb}[1]{\mathbf{#1}}
\newcommand{\dv}[2]{\frac{\d{#1}}{\d{#2}}}
\newcommand{\pdv}[2]{\frac{\partial{#1}}{\partial{#2}}}
\def\degree{$^{\circ}$}
\title{\textbf{实验一 \ce{SrO}-\ce{Al2O3}二元体系中几种荧光材料的合成和表征\cite{inorganic_chemistry_1}}}
\author{王崇斌\;1800011716}
\makeatletter
\makeatother
\begin{document}
	\pagestyle{fancy}
	\pagestyle{fancy}
    \lhead{无机化学实验}
	\chead{}
	\rhead{\today}
	\maketitle
    \thispagestyle{fancy}
	\section{实验目的}
	\begin{enumerate}
		\item 学习高温固相合成的基本方法
		\item 学习软化学制备前驱体的基本方法
		\item 了解固体荧光材料的发光原理和基本表征方法
	\end{enumerate}
	\section{实验原理}
	\subsection{高温固相合成}
	高温固相合成通常用于合成无机固体材料。反应中存在多个固体物相,反应主要
	发生在固态反应物的接触面上,同时在其上形成产物层;然后反应物通过扩散作用跨过产物层
	进一步反应。由于固相难以混合充分,反应物颗粒之间的接触面积大小受反应物颗粒直径影响明显
	,同时由于固体中原子扩散速率远低于液相或气相(晶格比较稳定,常温下只会在平衡位置
	附近振动),因此固相化学反应通常需要在高温下进行(增加固体中原子的扩散速率,甚至
	熔化反应物或者产物以达到充分混合的目的),反应时间较长,难以得到高纯度、均匀的、物相单一
	的产物。
	\par 
	通常影响固相化学反应速率的因素都有:反应温度(这个前面讨论了),反应物混合的均匀程度(决定了
	反应物之间的接触面积),反应物物相(不同的物相有着不同的表面能和稳定性),添加助溶剂(提高
	反应物表面离子的扩散速率),等等。
	\subsection{软化学制备前驱体}
	除了机械研磨之外,还可以针对不同的反应体系设计出利用化学反应来充分混合反应物的方法,这个
	过程称为软化学制备反应前驱体。其主要的思路是利用水溶液稳定均一的特性,从溶液中
	想办法“提取”出固态的前驱体,这就需要利用反应物的化学性质来设计制备前驱体的反应。
	通常使用的方法有\textbf{共沉淀法}、\textbf{溶胶凝胶法}、\textbf{燃烧法}。
	在本实验中使用了燃烧法。
	\par 共沉淀法是指在均相溶液中加入沉淀剂,使多种阳离子以\textbf{共晶}的方式沉淀下来,理想
	情况下共沉淀物可以达到原子量级的混合程度,混合效果远高于机械研磨法。
	但是由于不同物种溶度积不同,很有可能发生偏析——即某个物种优先沉淀的现象,
	从而导致混合并不均匀。因此沉淀剂的选择和沉淀条件(ph、温度等)的控制就显得尤为重要。
	溶胶凝胶法利用金属醇盐的缓慢水解,然后逐渐缩聚最终生成凝胶的过程将阳离子充分混合。燃烧法
	通常是在金属硝酸盐溶液中加入还原剂,加热浓缩,而后升高温度让硝酸根与还原剂发生激烈的氧化
	还原反应,可以在较短的时间内获得混合充分的金属氧化物混合物。通常加入的还原剂是含有羟基的
	羧酸比如柠檬酸或者甘氨酸和尿素这类能够起到一定螯合作用的还原剂。
	\subsection{荧光材料的发光原理}
	荧光材料一般由\textbf{基质}和\textbf{激活剂}组成。基质材料通常是由满壳层离子构成的稳定
	固体化合物,禁带较宽,在可见光范围内没有吸收;激活剂是不满壳层的离子,以固溶形式溶解在基
	质中,电子在壳层内或邻近壳层之间跃迁产生吸收和荧光发射。\ce{SrO-Al2O3}二元体系中几种
	不同Sr-Al比例的铝酸锶相均是优良的基质材料,\ce{Eu^{2+}}可以作为该基质中的优良激活剂。
	\ce{Eu^{2+}}的激发过程为$4f^7\to4f^ 65d ^1$的d-f跃迁,其中f电子构成受配位环境
	影响较小的窄能级,$5d$轨道受配位场影响较为明显。因此\ce{Eu^{2+}}周围的配位场越强,$d$
	轨道分裂越大,导致$d$轨道形成的能带下沿降低,因此基质的组成(在本实验中表现为\ce{Sr}与\ce{Al}
	)明显影响着其中\ce{Eu^{2+}}的发光行为。
	\par 
	如果进一步在这样的晶体中掺入\textbf{陷阱离子},有可能得到\textbf{长余辉材料}。\ce{Dy ^{3+}}
	是一种合适的陷阱离子,
	\section{实验操作步骤}
	\subsection{高温固相反应合成荧光材料}
	\subsection{燃烧法合成荧光材料前驱体}
	\section{荧光材料的表征}
	\subsection{发光性能的表征}
	\subsection{物相的鉴定}

	
	\bibliographystyle{plain}
	\bibliography{ref}
\end{document}